\documentclass[11pt]{article}

%A Few Useful Packages
% \usepackage{marvosym} 					%for loading fonts
\RequirePackage{color,graphicx}
\usepackage[usenames,dvipsnames]{xcolor}
% \usepackage[big]{layaureo} 				%better formatting of the A4 page
% an alternative to Layaureo can be ** \usepackage{fullpage} **
\usepackage{supertabular} 				%for Grades
\usepackage{titlesec}					%custom \section
\usepackage[hidelinks]{hyperref}
\usepackage{longtable}
\usepackage{url}
\usepackage{fullpage}
\usepackage{enumitem}

\usepackage[T1]{fontenc}
\usepackage{newtxtext,newtxmath}
% \usepackage{fourier}

% \usepac{}kage{setspace}
% \onehalfspacing

%Setup hyperref package, and colours for links
\usepackage{hyperref}
\definecolor{linkcolour}{rgb}{0,0.2,0.6}
\hypersetup{colorlinks,breaklinks,urlcolor=linkcolour, linkcolor=linkcolour}

%FONTS
%\defaultfontfeatures{Mapping=tex-text}
%\setmainfont[SmallCapsFont = Fontin SmallCaps]{Fontin}
%%% modified for Karol Kozioł for ShareLaTeX use

%%%

%CV Sections inspired by: 
\titleformat{\section}{\large\scshape\raggedright}{}{0em}{}[\titlerule]
\titlespacing{\section}{0pt}{3pt}{3pt}
%Tweak a bit the top margin
%\addtolength{\voffset}{-1.3cm}

%Italian hyphenation for the word: ''corporations''
\hyphenation{im-pre-se}

%-------------WATERMARK TEST [**not part of a CV**]---------------
\usepackage[absolute]{textpos}

\setlength{\TPHorizModule}{30mm}
\setlength{\TPVertModule}{\TPHorizModule}
\textblockorigin{2mm}{0.65\paperheight}
\setlength{\parindent}{0pt}

%--------------------BEGIN DOCUMENT----------------------
\begin{document}

%WATERMARK TEST [**not part of a CV**]---------------
%\font\wm=''Baskerville:color=787878'' at 8pt
%\font\wmweb=''Baskerville:color=FF1493'' at 8pt
%{\wm 
%	\begin{textblock}{1}(0,0)
%		\rotatebox{-90}{\parbox{500mm}{
%			Typeset by Alessandro Plasmati with \XeTeX\  \today\ for 
%			{\wmweb \href{http://www.aleplasmati.comuv.com}{aleplasmati.comuv.com}}
%		}
%	}
%	\end{textblock}
%}

% \pagestyle{empty} % non-numbered pages

%--------------------TITLE-------------
\par{
		{\LARGE {Chinmay Nirkhe}
	}\bigskip\par}

%--------------------SECTIONS-----------------------------------


\section{Current Employment}
\textbf{IBM Quantum} \\
Aug 2022 - current \\

\textit{Research Staff Member} \\
MIT-IBM Watson AI Lab \\
314 Main Street Cambridge, Mass. 02142


%Section: Education
\section{Education}

\textbf{University of California, Berkeley} ({\small \textsc{GPA}: 3.9/4.0}) \\
Aug 2017 - Dec 2022\\

\textit{Ph.D. Computer Science} \\
Advisor: Professor Umesh Vazirani \\ 
Thesis: \href{https://www2.eecs.berkeley.edu/Pubs/TechRpts/2022/EECS-2022-184.html}{Lower bounds on the complexity of quantum proofs} \\

\textbf{California Institute of Technology} ({\small \textsc{GPA}: 3.9/4.0}) \\
Sep 2013 - Jun 2017 \\

\textit{B.S. Mathematics}
% The degree involves year long courses in Algebra, Analysis, Topology, Discrete Mathematics, which I
% supplemented with elective courses in Probability Theory, Stochastic Process and Calculus, Markov
% Chains, and Computability Theory. \\

\textit{B.S. Computer Science}
% The degree involves courses in theory, algorithms, functional and objective programming, and software
% development, which I supplemented with electives courses in advanced algorithms, computer science
% research, machine learning, and networks.








\section{Publications and Preprints}
\textit{All works have alphabetical authorship by surname and signify equal contributions by all authors.}

\
\begin{enumerate}[{leftmargin=*,start=10,label=[\arabic*]\addtocounter{enumi}{-2}}]

\item \textbf{A classical oracle separation between QMA and QCMA}

\emph{Anand Natarajan and Chinmay Nirkhe.} 2022.

In submission.

\item \textbf{NLTS Hamiltonians from good quantum codes}

\emph{Anurag Anshu, Nikolas Breuckmann, and Chinmay Nirkhe.} 2022.

Publicly available pre-print: \href{https://arxiv.org/abs/2206.13228}{$\mathtt{arXiv:2206.13228}$}. \href{https://www.quantamagazine.org/computer-science-proof-lifts-limits-on-quantum-entanglement-20220718/}{News article} published by Quanta Magazine. \href{https://blog.simons.berkeley.edu/2022/08/the-blind-men-and-the-quantum-elephants/}{Simons institute blog post}.

\item \textbf{The parametrized complexity of quantum verification}

\emph{Srinivasan Arunachalam, Sergey Bravyi, Chinmay Nirkhe, and Bryan O'Gorman.} 2022.

In the proceedings of \href{https://tqc2022-conference.iquist.illinois.edu/}{17th Conference on the Theory of Quantum Computation, Communication and Cryptography (TQC 2022)}. Publicly available pre-print: \href{https://arxiv.org/abs/2202.08119}{$\mathtt{arXiv:2202.08119}$}.

\item \textbf{Quantum search-to-decision and the state synthesis problem}

\emph{Sandy Irani, Anand Natarajan, Chinmay Nirkhe, Sujit Rao, and Henry Yuen.} 2021.

Presented at \href{https://web.cvent.com/event/8adf8248-432b-499c-91e2-63b83ba3f69e/summary}{25th Annual Conference on Quantum Information Processing (QIP 2022).}
In the proceedings of \href{https://www.computationalcomplexity.org/}{37th Computational Complexity Conference (CCC 2022)}. Publicly available pre-print: \href{https://arxiv.org/abs/2111.02999}{$\mathtt{arXiv:2111.02999}$}.

\item \textbf{Circuit lower bounds for low-energy states of quantum code Hamiltonians}

\emph{Anurag Anshu and Chinmay Nirkhe.} 2021.

To appear in Theory of Computing 2022. In the proceedings of \href{http://itcs-conf.org/itcs22/itcs22-cfp.html}{13th Innovations in Theoretical Computer Science Conference (ITCS 2022).}
Presented at \href{https://www.mcqst.de/qip2021/}{24th Annual Conference on Quantum Information Processing (QIP 2021).} Publicly available pre-print: \href{https://arxiv.org/abs/2011.02044}{$\mathtt{arXiv:2011.02044}$}.

\item \textbf{Good approximate quantum LDPC codes from spacetime circuit Hamiltonians}

\emph{Thomas Bohdanowicz, Elizabeth Crosson, Chinmay Nirkhe, and Henry Yuen.} 2019.

In the proceedings of \href{http://acm-stoc.org/stoc2019/}{51st ACM Symposium on the Theory of Computation (STOC 2019).}
Presented at \href{http://qec19.iopconfs.org/home}{5th International Conference on Quantum Error Correction (QEC 2019).} (Invited Talk)
Presented at \href{https://jila.colorado.edu/qip2019/}{22nd Annual Conference on Quantum Information Processing (QIP 2019).}
Publicly available pre-print: \href{https://arxiv.org/abs/1811.00277}{$\mathtt{arXiv:1811.00277}$}.

\item \textbf{On the complexity and verification of quantum random circuit sampling}

\emph{Adam Bouland, Bill Fefferman, Chinmay Nirkhe, and Umesh Vazirani.} 2018.

Published in \href{https://www.nature.com/articles/s41567-018-0318-2}{Nature Physics 2018.}
Presented at \href{http://itcs-conf.org/itcs19/itcs19-cfp.html}{2020 ACM Conference on Innovations in Theoretical Computer Science (ITCS 2019).}
Presented at \href{https://jila.colorado.edu/qip2019/}{22nd Annual Conference on Quantum Information Processing (QIP 2019).}
Publicly available pre-print: \href{https://arxiv.org/abs/1803.04402}{$\mathtt{arXiv:1803.04402}$}.
\href{https://news.berkeley.edu/2018/10/29/berkeley-computer-theorists-show-path-to-verifying-that-quantum-beats-classical/}{News article} published by UC Berkeley News.

\item \textbf{Approximate low-weight check codes and circuit lower bounds for noisy ground states}

\emph{Chinmay Nirkhe, Umesh Vazirani, and Henry Yuen.} 2018.

In the proceedings of \href{http://drops.dagstuhl.de/opus/volltexte/2018/9095/}{45th International Colloquium on Automata, Languages, and Programming (ICALP 2018).} 
Presented at \href{https://www.tqc2018.org/}{13th Conference on the Theory of Quantum Computation, Communication, and Cryptography (TQC 2018).}
Publicly available pre-print: \href{https://arxiv.org/abs/1802.07419}{$\mathtt{arXiv:1802.07419}$}.

\end{enumerate}


\section{Talks \& Presentations}

\textbf{A classical oracle separation between QMA and QCMA}

Harvard Quantum Information Group Meeting, October 2022.

\

\textbf{NLTS Hamiltonians from good quantum codes}

QLA Meets QIT II Workshop, Chicago, Illinois, USA, November 2022.

INTRIQ Biannual Meeting, Montreal, Quebec, Canada, October 2022.

Columbia University Seminar, October 2022.

CIQC Seminar, UC Berkeley, August 2022.

My PhD dissertation, UC Berkeley, August 2022.

University of Texas, Austin, September 2022.

Kavli Institute for Theoretical Physics, UCSB, September 2022.

Stanford University Institute for Theoretical Physics, August 2022.

University of Sydney, July 2022.

IBM Quantum Research Group, July 2022.

\

\textbf{The parametrized complexity of quantum verification}

Theory of Quantum Computation, July 2022.

University of Maryland QuICS Seminar, June 2022.

IBM Quantum Research Group, September 2021.

\

\textbf{Quantum search-to-decision and the state synthesis problem}

Computational Complexity Conference, July 2022.

University of Maryland QuICS Seminar, June 2022.

UC Berkeley Theory Lunch Seminar, April 2022.

Quantum Information Processing, March 2022.

University of Texas, Austin Scott Aaronson's Group Quantum Seminar, December 2021.

\

\textbf{Circuit lower bounds for low-energy states of quantum code Hamiltonians}

Innovations in Theoretical Computer Science, February 2022.

Simons Institute, Quantum Wave in Computing Reunion Workshop, July 2021.

Stanford Patrick Hayden's Group Quantum Seminar, July 2021.

Caltech Thomas Vidick's Group Quantum Seminar, March 2021.

Quantum Information Processing (QIP), February, 2021.

University of Texas, Austin and MIT Joint Quantum Seminar, December 2020. 

Quantum Code Design and Architectures (The European Network) Seminar, November 2020.

\

\textbf{Good approximate QLDPC codes from spacetime Hamiltonians}

Symposium on the Theory of Computing, June 2019.

Institute for Quantum Computation, University of Waterloo Seminar, April 2019.

UC Berkeley Theory Lunch, February 2019.

Berkeley Quantum Information \& Computation Center Seminar, December 2018.

\

\textbf{On the complexity and verification of Random Circuit Sampling}

Indian Symposium on Quantum Information and Technology, Pune, India, December 2019.

University of Toronto Computer Science / Quantum Information Seminar, March 2019.

IIT Kanpur Computer Science Seminar, December 2018.

Simons Institute Industry Day Lightning Talks, May 2018.

UC Berkeley Visit Days, March 2018.

\

\textbf{Approximate low-weight check codes and circuit lower bounds for noisy ground states}

Theory of Quantum Computing, July 2018.

Institute for Theoretical Physics Seminar, May 2018.

Caltech IQIM, February 2018.

\

\textbf{Other talks}

UC Berkeley Quantum Brainstorming Session, December 2021.

% \

% \textbf{Quantum pseudo-telepathy games}

% Caltech undergraduate math club, November 2016.




\section{Past Employment and Research Positions}

\textbf{IBM Quantum.}
San Jose, California and Yorktown Heights, New York. PhD Quantum Research Intern. May - December 2021.

\

\textbf{Jane Street Capital.}
New York City, NY. Software Engineer Intern. Summer 2016. 

\

\textbf{Caltech Research for Course Credit, Professor Thomas Vidick.}
Summer and Fall 2016. Theoretical computer science research on pseudo-telepathy quantum games and certifiable randomness generation. 

\

\textbf{Twitter, Inc.}
San Francisco, CA. Software Engineer Intern. Summer 2015. 

\	

\textbf{Caltech Summer Undergraduate Research Fellowship, Professor Thomas Apostol.}
Summer 2014. Mathematics research on the geometry of brachistochrone and tautochrones in radially dependent force fields. 

\

\textbf{University of Washington, Professor Jacob O. Wobbrock.}
Fall 2012. Human computer interaction research on novel text entry systems using \emph{Microsoft Kinect} for midair freehand gestural input. Unpublished publication: C. Nirkhe, J. Wobbrock; \emph{The Bubble Keyboard: A Midair Freehand Gestural Text Entry Method.}

\section{Awards}
UC Berkeley EECS Nominee for the Microsoft Research PhD Fellowship 2019

National Science Foundation Graduate Research Fellowship Honourable Mention 2017 

Microsoft Teaching in Computational Mathematical Sciences (CMS) Prize 2017

Associated Students of the California Institute of Technology (ASCIT) Teaching Award 2017 

National Merit Semifinalist 2012 \\

\section{Teaching Positions}

\textbf{Quantum Interactive Protocols}

Lecturer; University of California, Berkeley Spring 2022.

\textbf{The Mathematics of Quantum Computation}

Teaching Assistant; Hebrew University of Jerusalem: The 4th Winter School in Computer Science and Engineering Fall 2019.

\textbf{CS 294-6: Quantum Computation}

Teaching Assistant; University of California, Berkeley Fall 2019.

\textbf{Trends in Theory: Quantum Computation}

Teaching Assistant; University of California, San Diego Spring 2018.

\textbf{CS 170: Efficient Algorithms and Intractable Problems}

Teaching Assistant; University of California, Berkeley Spring 2018.

\textbf{CS 38: Introduction to Algorithms}

Head Teaching Assistant; California Institute of Technology Spring 2016, Spring 2017.

\textbf{CS 139: Advanced Algorithms}

Teaching Assistant; California Institute of Technology Winter 2017.

\textbf{CS 156a: Learning Systems}

Teaching Assistant; California Institute of Technology Fall 2016.

\textbf{CS 21: Decidability and Tractability}

Teaching Assistant; California Institute of Technology Winter 2016.

% \section{Leadership}
% CMS (Computational and Mathematical Sciences) Undergraduate Advisory Council Member 2018

% Ricketts Hovse (Caltech Dormitory) Social Director 2017

% Ricketts Hovse Athletics Manager 2016-18 \\

\section{Mentorship}
\textbf{Jyoti Rani}
Spring 2022, 
UC Berkeley undergraduate.

\textbf{Samyak Surti}
Spring \& Summer 2022,
UC Berkeley undergraduate.

\textbf{James Chen}
Summer \& Fall 2020,
UC Berkeley undergraduate, now quantitative trading at Jane Street Capital.

\textbf{Natalie Parham}
Summer \& Fall 2020,
UC Berkeley undergraduate, now a Ph.D. student at Columbia University.

\textbf{Sahil Patel} 
Summer \& Fall 2020,
UC Berkeley undergraduate, now a Masters student at UC Berkeley.

\section{Service}
Program committee member for QIP 2023.

(Anonymous) reviewer for\footnote{Parentheticals denote multiplicity.} 
\begin{itemize}
\item (Conferences) CCC 2018, RANDOM 2018, TCC 2018, TQC 2018, CCC 2019, QCRYPT 2019, QIP 2019 (2), QIP 2020 (4), STOC 2020 (2), FOCS 2020, SODA 2021, ITCS 2021, STOC 2021 (3), TQC 2021, ITCS 2022 (2), QIP 2022 (3), STOC 2022 (2) and FOCS 2022 (2).
\item (Journals) Quantum (3).
\end{itemize}


\section{References}
Professor Umesh Vazirani, University of California, Berkeley. \texttt{vazirani@cs.berkeley.edu}.

Assistant Professor Henry Yuen, Columbia University. \texttt{henry.yuen@columbia.edu}.

Assistant Professor Anurag Anshu, Harvard University. \texttt{anuraganshu@seas.harvard.edu}.

%Section: Personal Data
\section{Personal Data}

\begin{tabular}{rl}
    \textsc{Address:}     & IBM Research, Attn: Chinmay Nirkhe, 314 Main St Cambridge, MA 02142 \\
    % \textsc{Personal Address:}   & 1634 Oxford St Apt 303 Berkeley, California 94709 \\
    % \textsc{Phone:}       & +1 (425) 444 6586\\
    \textsc{Email:}       & $\mathtt{nirkhe@ibm.com}$ \\
    \textsc{Website:}     & \url{https://cs.berkeley.edu/~nirkhe} \\
    \textsc{Nationality:} & United States of America \\
    \textsc{Pronouns:}    & he/him \\
    % \textsc{Birthday:}       & 5th of July, 1995
\end{tabular}

\

\

\textsc{Last updated: \today}

\end{document}
